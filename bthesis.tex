\documentclass[12pt,a4paper,twoside]{report}

% link to wordpage source \href{https://campuscvut-my.sharepoint.com/:w:/g/personal/hasplvoj_cvut_cz/EY8pybo8CfdNmaBXNMXVtx0BKKOaEe7kFf99Fhh3AoGY8Q?e=3V3gST}{\color{blue}{wordpage}}

\usepackage{packages/cvut}
% Graphics
\usepackage[colorlinks=true,linkcolor=black,anchorcolor=black,citecolor=black,filecolor=black,menucolor=black,runcolor=black,urlcolor=black]{hyperref}
\usepackage[labelsep=period]{caption}
%\usepackage{subcaption}
\usepackage{graphicx}
\usepackage{pdfpages}
\usepackage{subfig}
\usepackage{float}
\usepackage{caption}

% Math
\usepackage{amsthm}
\usepackage{amsmath}
\usepackage{amsfonts}

% Lists
\usepackage{enumerate}
\usepackage{paralist}
\usepackage{acronym}
\usepackage[nonumberlist]{glossaries} 

% Tables
\usepackage{booktabs}
\usepackage{multirow}
\usepackage{multicol}
\usepackage{tabularx}
\usepackage{etoolbox}

% Algorithms
\usepackage{algorithm}
\usepackage{algpseudocode}

% Title etc.
\usepackage{chngcntr}
\usepackage{cite}
\usepackage{titlesec}

% etc.
\usepackage{setspace}
\usepackage{color}
\usepackage{pdflscape}
\usepackage{afterpage}
\usepackage[nottoc]{tocbibind} 
\usepackage[toc]{appendix}
\usepackage{listings}
\usepackage{xcolor}
\usepackage[usestackEOL]{stackengine}
\usepackage{rotating}

%\usepackage[resetlabels,labeled]{multibib}
%\usepackage[T1]{fontenc}
%\usepackage{chngcntr}
%\counterwithin{Figure}{chapter}

%------------------
%  define COLOR
%------------------
\definecolor{darkgreen}{rgb}{0,0.4,0}
\definecolor{darkorange}{RGB}{211, 84, 0}


% TODO pridat article jako zdroj globaldata do INTRA

\setlength{\parindent}{0pt}                 % creates normal paragrapsh separation
\setlength{\parskip}{10pt}                  % creates normal paragraph separation

\graphicspath{ {./images/} }                % where to look for images in \includegraphics{xxx.png} command

\setacronymstyle{long-sc-short}
% \usepackage{subfigure}
 \setacronymstyle{long-short}

%\counterwithout{figure}{chapter}
%\counterwithout{equation}{section} % undo numbering system provided by phstyle.cls
%\counterwithout{equation}{chapter} % implement desired numbering system

 \newtheorem{my_theorem}{Theorem} 
 \newtheorem{Assumption}{\bf AS}
 \newtheorem{my_lemma}{Lemma}
 \newtheorem{my_prep}{Proposition}
 \newtheorem{my_remark}{Remark}
 \newtheorem{my_corr}{Corollary}
 
 \DeclareMathOperator*{\argmin}{\arg\!\min}
 \DeclareMathOperator*{\argmax}{\arg\!\max}
 
 
 \definecolor{GreenTable}{cmyk}{0.59,0,0.88,0.27}
 
 	\providecommand{\keywords}[1]{\vspace{4pt}\textbf{\textit{Keywords: }} #1}
 	\providecommand{\keywordscz}[1]{\vspace{4pt}\textbf{\textit{Klíčová slova: }} #1}
 	
% \pdfminorversion=7
% \makeatletter
% \def\bstctlcite{\@ifnextchar[{\@bstctlcite}{\@bstctlcite[@auxout]}}
% \def\@bstctlcite[#1]#2{\@bsphack
	
% 	\@for\@citeb:=#2\do{%
% 		\edef\@citeb{\expandafter\@firstofone\@citeb}%
% 		\if@filesw\immediate\write\csname
% 		#1\endcsname{\string\citation{\@citeb}}\fi}%
% 	\@esphack}
% \makeatother

\DeclareRobustCommand{\ttfamily}{\fontencoding{T1}\fontfamily{lmtt}\selectfont}

%\usepackage{titlesec}
% defines things like author and date or field of study for titlepage
\DeclareMathOperator*{\argmaxA}{arg\,max}
\DeclareMathOperator*{\argminA}{arg\,min} 
\title {Development of Testbed for Vehicular Edge Computing}
\author{Vojtěch Hašpl}


\date{My 2023}

\argument{supervisor}{Ing. Jan Plachý, Ph.D.}
\argument{cosupervisor}{}
\argument{department}{Department of Telecommunication Engineering}
\argument{programme}{Electronics and Communications}
\argument{specialization}{-}
\argument{dateyear}{2023}
\argument{datemonth}{May}
\argument{location}{Prague}
\argument{fulldate}{May,~2023}
\argument{fulldatecz}{25. května~2023}
 \renewcommand{\chaptername}{}

%\addto\captionsczech{\renewcommand{\keywordsname}{Klíčová slova}} 
 \makeatletter
 	
% \def\@makechapterhead#1{%
% \vspace*{20\p@}%
% {\parindent \z@ \raggedright \normalfont
% %\ifnum \c@secnumdepth >\m@ne
% % \huge\bfseries \@chapapp\space \thechapter
% % \par\nobreak
% % \vskip 20\p@
% %\fi
% \interlinepenalty\@M
% \Huge \bfseries \thechapter. \Huge \bfseries #1\par\nobreak
% \vskip 40\p@
% }}
\titleformat{\chapter}{\huge\bfseries}{\thechapter.}{20pt}{\huge\bfseries}
 \makeatother
% \renewcommand{\bibname}{References}
 

 \newenvironment{spodnitext}[1]{
 	\cleardoublepage
 	\null
 	\vfill
 	\section*{#1}
 	}{
 	\vspace{10mm}
 	}
 	
 
% 	\renewcommand{\bibname}{References} 
 
 
 \AtBeginDocument{\renewcommand{\bibname}{References}}	
 \def\@makechapterhead#1{%
 	%%%%\vspace*{50\p@}% %%% removed!
 	{\parindent \z@ \raggedright \normalfont
 		\ifnum \c@secnumdepth >\m@ne
 		\huge\bfseries \@chapapp\space \thechapter
 		\par\nobreak
 		\vskip 20\p@
 		\fi
 		\interlinepenalty\@M
 		\Huge \bfseries #1\par\nobreak
 		\vskip 20\p@
 	}}
 	\def\@makeschapterhead#1{%
 		%%%%%\vspace*{50\p@}% %%% removed!
 		{\parindent \z@ \raggedright
 			\normalfont
 			\interlinepenalty\@M
 			\Huge \bfseries #1\par\nobreak
 			\vskip 1\p@
 		}}

\renewcommand*{\glsgroupskip}{}

% -----------------------
%  listings definitions
% -----------------------
\lstnewenvironment{flask}[1][]
{
    \lstset{
        language=Python,
        frame=single,
        basicstyle=\small\ttfamily,
        keywordstyle=\bfseries\color{blue},
        stringstyle=\color{darkorange},
        commentstyle=\color{darkgreen},
        showstringspaces=false,
		tabsize=4,
        captionpos=b,
        numbers=left,
        numbersep=5pt,
        #1,
		breaklines,
		float
    }
}{}

% 	
%\bibliographystyle{IEEEtran}
%\bibliography{IEEEabrv,minimum}
%\bibliographystyle{ieeetr}	
%\makenoidxglossaries

% ---------------------
%   Listing Acronyms
% ---------------------

\newacronym{iot}{IoT}{Internet of Things}
\newacronym{sdo}{SDO}{Standard Defining Organization}
\newacronym{3gpp}{3GPP}{Third Generation Partnership Project}
\newacronym{etsi}{ETSI}{Eurupean Telecommunication Standard Institution}
\newacronym{o-ran}{O-RAN}{Open Radio Access Network}
\newacronym{sms}{SMS}{Short Message Service}
\newacronym{vr}{VR}{Virtual Reality}
\newacronym{ran}{RAN}{Radio Access Network}
\newacronym{nr}{NR}{New Radio}
\newacronym{sba}{SBA}{Service Based Architecture}
\newacronym{ec}{EC}{Edge Computing}
\newacronym{cc}{CC}{Cloud Computing}
\newacronym{avs}{AVs}{Autonomous Vehicles}
\newacronym{mec}{MEC}{Multi-Access Edge Computing}
\newacronym{isg}{ISG}{Industry Specification Group}
\newacronym{vec}{VEC}{Vehicular Edge Computing}
\newacronym{b2b}{B2B}{Business-to-Business}
\newacronym{embb}{eMBB}{Enhanced Mobile Broadband}
\newacronym{mmtc}{mMTC}{Massive Machine Type Communication}
\newacronym{urllc}{URLLC}{Ultra Reliable Low Latency Communication}
\newacronym{nfv}{NFV}{Network Function Virtualization}
\newacronym{sdn}{SDN}{Software Defined Networking}
\newacronym{cups}{CUPS}{Control and User Plane Separation}
\newacronym{epc}{EPC}{Evolved Packet Core}
\newacronym{nfs}{NF}{Network Functions}
\newacronym{apis}{API}{Application Programmable Interface}
\newacronym{http}{HTTP}{Hypertext Transfer Protocol}
\newacronym{rest}{REST}{Representational State Transfer}
\newacronym{ue}{UE}{User Equipment}
\newacronym{upf}{UPF}{User Plane Function}
\newacronym{dn}{DN}{Data Network}
\newacronym{cn}{CN}{Core Network}
\newacronym{nssf}{NSSF}{Network Slice Selection Function}
\newacronym{nef}{NEF}{Network Exposure Funciton}
\newacronym{ausf}{AUSF}{Authentication Server Function}
\newacronym{nrf}{NRF}{Network Repository Function}
\newacronym{amf}{AMF}{Access and Mobility Management Function}
\newacronym{pcf}{PCF}{Policy Control Function}
\newacronym{smf}{SMF}{Session Management Function}
\newacronym{udm}{UDM}{Unified Data Management}
\newacronym{af}{AF}{Application Funciton}
\newacronym{pdu}{PDU}{Packet Data Unit}
\newacronym{qos}{QoS}{Quality of Service}
\newacronym{rrc}{RRC}{Radio Resource Control}
\newacronym{ul cl}{UL CL}{Uplink Classifier}
\newacronym{fes}{FEs}{Functional Entities}
\newacronym{mec apps}{MEC apps}{MEC Applications}
\newacronym{mp}{Mp}{MEC platform functionality reference point}
\newacronym{mm}{Mm}{MEC Management reference point}
\newacronym{mx}{Mx}{Reference point connecting to external entities}
\newacronym{oss}{OSS}{Operations Support System}
\newacronym{ualcmp}{UALCMP}{User Application Lifecycle Management Proxy}
\newacronym{meo}{MEO}{MEC Orchestrator}
\newacronym{mepm}{MEPM}{MEC Platform Manager}
\newacronym{vim}{VIM}{Virtualization Infrastructure Manager}
\newacronym{vm}{VM}{Virtual Machine}
\newacronym{k8s}{k8s}{Kubernetes}
\newacronym{cvb}{CVB}{Connected Vehicles Blueprint}
\newacronym{ealtedge}{EALTEdge}{Enterprise Application on Lightweight Telco Edge}
\newacronym{pcei}{PCEI}{Public Cloud Edge Interface}
\newacronym{openness}{OpenNESS}{Open Network Edge Services Software}
\newacronym{oai}{OAI}{OpenAirInterface}
\newacronym{osa}{OSA}{OpenAirInterface Software Alliance}
\newacronym{ll-mec}{LL-MEC}{Low-Latency MEC}
\newacronym{mep}{MEP}{MEC Platform}
\newacronym{rnis}{RNIS}{Radio Network Information Service}
\newacronym{enb}{eNB}{evolved NodeB}
\newacronym{gnb}{gNB}{new generation NodeB}
\newacronym{tcp}{TCP}{Transmission Control Protocol}
\newacronym{ip}{IP}{Internet Protocol}
\newacronym{sdap}{SDAP}{Service Data Application Protocol}
\newacronym{pdcp}{PDCP}{Packet Data Convergence Protocol}
\newacronym{rlc}{RLC}{Radio Link Control}
\newacronym{mac}{MAC}{Media Access Control}
\newacronym{phy}{PHY}{Physical Layer}
\newacronym{nas}{NAS}{Non-Access Stratum}
\newacronym{rtt}{RTT}{Rount Trip Time}
\newacronym{tdd}{TDD}{Time Division Duplex}
\newacronym{fdd}{FDD}{Frequency Division Duplex}
\newacronym{nfapi}{nFAPI}{Network Functional API}
\newacronym{cots}{COTS}{Customer Off-The-Shelf}
\newacronym{udr}{UDR}{Unified Data Repository}
\newacronym{flexric}{FlexRIC}{Flexible RAN Intelligent Controller}
\newacronym{flexcn}{FlexCN}{Flexible Core Controller}
\newacronym{oai-cm}{OAI-CM}{OpenAirInterface Configuration Manager}
\newacronym{kpi}{KPI}{Key Performance Indicator}
\newacronym{cli}{CLI}{Command Line Interface}
\newacronym{gui}{GUI}{Graphical User Interface}


\makeglossaries

% capton setup
%\captionsetup[figure]{name=Figure}

\setstretch{1.15}

% -----------------------
% THE BEGIN DOCUMENT HERE
% -----------------------
\begin{document}

% -----------------------
%        Titlepage
% -----------------------
\maketitle

% -----------------------
%      Assignment
% -----------------------
\includepdf{assignment/assignment_eng.pdf}


% -----------------------
%      Statement
% -----------------------
\makestatement


%\pagestyle{empty}
\pagenumbering{roman}


% -----------------------
%    Acknowledgement
% -----------------------
\chapter*{Acknowledgements}
	% \addcontentsline{toc}{chapter}{Acknowledgments}
	Thank you


% -----------------------
%        Abstracts
% -----------------------
\chapter*{Abstract}
Recent advancements in Autonomous Vehicles (AVs) research are bringing AVs closer to widespread adoption. However, the computational requirements of AVs demand robust hardware resources, resulting in substantial power consumption for the vehicles. This has led to the emergence of Vehicular Edge Computing (VEC), a research area focused on utilization of processing resources at mobile networks’ edge for offloading computationally demanding tasks from vehicles at low latency. To enable authentic simulation for evaluating VEC algorithms, a testbed leveraging Multi-Access Edge Computing (MEC) is proposed in this thesis. The set up testbed is based on MEC platform by OpenAirInterface (OAI) and the underlying OAI 5G mobile network, which facilitates accurate and scalable simulations. In evaluation against cloud deployed 400 kilometers away, the testbed achieves approximately 59 \% lower Round Trip Time (RTT) demonstrating its viability for realistic evaluation of VEC algorithms.
	
\keywords{vehicular edge computing, multi-access edge computing, mobile networks, 5G, task offloading, autonomous vehicles}
%
%fill in abstract
%
\newpage

\chapter*{Abstrakt}
	% \addcontentsline{toc}{chapter}{Abstrakt}	% comment to omit abstract from toc
\keywordscz{vehicular edge computing, multi-access edge computing, mobilní sítě, 5G, přesun výpočtů, autonomn\'i vozidla}
%
%fill in abstract
%
\newpage

% -----------------------
%        TOC
% -----------------------
\tableofcontents			

% -----------------------
%  GLOSSARIES (ABBREV)
% -----------------------
\glsaddallunused
\printglossary[type=\acronymtype,title={List of Abbreviations}]	% prints glossaries alphabetically

% -----------------------
%   List of Figures- WIP
% -----------------------


%\newpage
%\thispagestyle{empty}
%\mbox{}
\newpage

\setcounter{page}{1}
%\end{otherlanguage}
\pagenumbering{arabic}

\setstretch{1.15}
%-----------------------
%  Ch1- Introduction
%-----------------------
\chapter{Introduction}
Mobile networks provide connectivity to mobile users, but through their recent development they are seen as a key enabler for emerging services and applications in areas such as Internet of Things (IoT) or automotive industry, whose requirements on communication data rate and latency are ever increasing. This drives the need for innovation and development of mobile networks. The development of mobile networks follows standards defined by multiple standards defining organizations (SDO). One of the SDOs is the Third Generation Partnership Project (3GPP), which unites seven telecommunications standard development organizations and their members, that define exact standards for all aspects of mobile networks such as architecture, protocols, interfaces, modulation techniques etc. The European SDO taking part in 3GPP is European Telecommunications Standard Institution (ETSI). Another noteworthy body involved in mobile network development is the Open Radio Access Network (O-RAN) whose aim is to promote development of open specifications for RAN components \cite{o-ran-web}. 

Development of mobile networks had started in 1980s with the first generation of mobile networks (1G) that focused on analog communication and enabled voice calls \cite{sauter2017history}. The 1G was followed by 2G which replaced the analog communication with digital communication and introduced short messages service (SMS) \cite{sauter2017history}. 3G, as a successor, has mainly focused on enhancing data rates to enable multimedia applications such as video streaming \cite{sauter2017history}. 4G then disrupted mobile networks design by switching to fully IP based architecture and further enhancing data rates and decreasing communication latency \cite{sauter2017history,dahlman-2013-4g}. The 4G is followed by the 5G which is focused on increasing the number of communicating devices, as well as further reducing communication latency. 

The initial deployment of 5G started in 2019 and is still ongoing. 5G has been developed with the aim of providing support not only for fast internet access, but also massive machine-type communication and mission critical communication \cite{dahlman-2020-5g}. The 5G is a flexible mobile network, capable of supporting novel applications and services, such as virtual reality (VR), IoT or connected vehicles. The development of the mobile network from 4G is seen both in the Radio Access Network (RAN) with introduction of 5G New Radio (NR) interface and in the core network (CN) by evolving the 4G CN into a Service Based Architecture (SBA) that provides as a flexible platform, capable of integrating novel applications and techniques like edge computing \cite{sabella-mec-sw-dev}. 

Edge computing (EC) can be described as an extension of Cloud Computing (CC) to the network edge \cite{sabella-mec-sw-dev}. CC is a computing paradigm that provides mobile network users’ access to a pool of resources, e.g. (storage, computing, applications and services), through the network in similar fashion to the wired networks \cite{mell2011nistCC}. The EC brings the resources closer to the user, to the so-called network edge to offer services with low latency. The low latency is critical for applications such as autonomous vehicles (Avs). An EC implemented in the mobile networks is called Multi-Access Edge Computing (MEC), which has been formerly known as Mobile Edge Computing. MEC is a RAN access agnostic, supporting mobile as well as Wi-Fi networks. The leading international SDO focused on MEC is ETSI that develops ETSI MEC standards within Industry Specification Group (ISG). The aim of ETSI MEC is to create an open standardized environment allowing for seamless integration of applications from vendors, service providers as well as third parties \cite{mec-etsi-web}. In this thesis ETSI MEC standards developed are followed to achieve full compliance with standardized networks. 

Through offering sizeable computational resources available at low latency, MEC enhances various use cases such as IoT, augmented reality, location services or Avs by providing high powered computing capabilities to computation limited devices \cite{mec-etsi-web}. AVs require processing of large amounts of data in order to recognize the surrounding environment and decide the next best action (slow down, speed up, turn etc.). This is a computationally demanding process, posing a computation challenge to the local (vehicle) limited computation capacity. As shown in recent research \cite{zhao2019towards}, the computation required for level-4 and level-5 Avs requires high performance server grade hardware. This is where MEC can be leveraged to offload some of the data processing from the AV to the network edge. This enables to drive down the vehicles’ cost and power consumption, while keeping the communication latency low. The utilization of MEC for vehicular applications is referred to as vehicular edge computing (VEC). 

As of today, VEC development is still in progress. To exploit computation offloading by vehicles with sufficiently low latency, algorithms for allocation of computation and communication resources need to be developed and evaluated. Therefore, this thesis focuses on developing a functional testbed for VEC. The structure of the thesis is as follows: first, integration of MEC in 5G networks is explained. Then, in Chapter~\ref{Ch:o-sMEC}, several existing open-source solutions for emulating network edge computing are examined and a suitable platform is selected for integration with emulated 5G mobile network. Chapter~\ref{Ch:OAI} provides a description of OpenAirInterface for emulation of mobile networks. Chapter~\ref{Ch:VEC-tstbed} is dedicated towards setting up the VEC testbed and its evaluation against reasonably distant cloud computing. Finally, in the last chapter, the thesis is concluded, and future work outlined.
% PRIDAT reference na kapitoly
%-----------------------
%  Ch2- MEC & 5G
%-----------------------
\chapter{MEC \& 5G}
\label{Ch:MECn5G}
The aim of this chapter is to provide a detailed description of MEC integration into 5G mobile networks. Although it is possible to integrate MEC into 4G, as described by ETSI \cite{ETSI:wp24}, MEC has been standardized after 4G networks have already been deployed. Therefore, MEC is mostly deployed with 5G, which enables seamless integration \cite{ETSI:wp28} and provides lower latency \cite{dahlman-2020-5g}, a crucial parameter for edge applications such as Avs.  In this chapter, both 5G and MEC system architectures are described, followed by a framework of MEC integration into 5G mobile network.

\section{Key concepts in 5G}
The evolution of mobile networks from 4G to 5G has focused not only on providing improved experience for end users, through enhancing communication data rates and decreasing communication latency, but also on introducing novel industry business opportunities, including Business to Business (B2B) \cite{rommer20195g}. To achieve that, two new industry-oriented communication services have been introduced to complement the end user oriented one – Enhanced Mobile Broadband (eMBB). The newly introduced services are Massive Machine-Type Communication (mMTC) and Ultra-Reliable Low Latency Communication (URLLC) \cite{rommer20195g}. Requirements of the three mentioned communication services are illustrated in Figure~\ref{F:usage-scenarios}. 
\begin{figure}[ht]
	\centering
	\includegraphics[width=\textwidth]{./images/usage-scenarios.png}
	\caption{Types of services in 5G \cite{dahlman-2020-5g}.}
	\label{F:usage-scenarios}
  \end{figure}

As shown in Figure~\ref{F:usage-scenarios}, the three illustrated usage scenarios for 5G vary largely in requirements, such as user experienced data rate, connection density or latency. Trying to cater for all three with different service requirements, at the same time, would be very inefficient. Therefore, the concept of network slicing has been introduced \cite{yousaf2017nfv}. Network slicing works by creating multiple, separated logical end-to-end networks on the same physical infrastructure \cite{yousaf2017nfv}. Each of these logical networks can then be orchestrated individually, according to the specific service requirements \cite{yousaf2017nfv}. In the context of Avs, this would allow for operation of Avs tailored network slice, providing low latency-oriented service.

The concept of network slicing requires innovative approaches for managing network resources. The technologies identified as enablers for network slicing are Network Function Virtualization (NFV) and Software Defined Networking (SDN) \cite{yousaf2017nfv}. NFV aims to leverage virtualization techniques to improve mobile networks’ flexibility, agility and scalability \cite{yousaf2017nfv}. SDN, on the other hand, aims to provide programmability to mobile networks’ services and software-based flow control of user data \cite{yousaf2017nfv}. 
  
Another important concept, although not new in 5G, is Control and User Plane Separation (CUPS). CUPS allows for distributed CN deployments, where the user plane endpoint can be deployed closer to the base station (and also end user) to reduce latency and save core network resources. The user plane solution for both 5G CN and 4G CN is similar, but the control plane underwent a revolution through introduction of SBA cite{rommer20195g}. The 5G CN Network Functions (NFs) are not connected with point-to-point interfaces, but rather to a common communication bus. Each NF exposes its services to other NFs via Application Programmable Interfaces (APIs) \cite{rommer20195g}, following the development trend towards exposing service functionalities via APIs. In practice, this is achieved through providing Hyper Text Transfer Protocol (HTTP) Representational State Transfer (REST) APIs \cite{sabella-mec-sw-dev,rommer20195g}. HTTP is a request response communication model protocol, while REST is an architecture style for creating flexible systems that can communicate with each other, using HTTP methods GET, POST, PUT and DELETE.

\section{5G system architecture}
This section introduces and describes the 5G system architecture with focus on NFs playing key role in MEC integration. As described earlier, 5G CN control plane NFs utilize SBA, which means that they are all logically interconnected. The user plane is connected to the rest of the network using point-to-point interfaces, as shown in the 5G system architecture \cite{ETSI:TS:5G}, depicted in Figure~\ref{F:5G-arch}. 

\begin{figure}[ht]
	\centering
	\includegraphics[width=\textwidth]{./images/5G-CN.png}
	\caption{5G system architecture \cite{ETSI:TS:5G}.}
	\label{F:5G-arch}
\end{figure}

The user plane consists of Radio Access Network (RAN), User Plane Function (UPF) and Data Networks Function (DN). The RAN itself consists of User Equipment (UE) and a base station – in 5G called gNodeB (gNB). It is worth noting that 5G is not designed exclusively for cellular radio access technologies, but instead aims to be access agnostic. The UE is connected to the gNB via the Uu interface, while the gNB itself is connected to UPF via the N2 interface. The main role of UPF is to process and forward user data. It also connects UEs with IP networks. UPF can be deployed closer to the end users, which reduces communication latency. This is crucial for MEC deployment, because UPF can be deployed at the network edge with the edge server, while UPF steers traffic directly to MEC. This placement of the UPF significantly reduces latency, in comparison to 4G where the traffic always has to be steered through the CN. \cite{rommer20195g}

The control plane consists of Network Slice Selection Function (NSSF), Network Exposure Function (NEF), Authentication Server Function (AUSF), Network Repository Function (NRF), Access and Mobility Management Function (AMF), Policy Control Function (PCF), Session Management Function (SMF), Unified Data Management (UDM) and Application Function (AF). The control plane NFs directly interacting with user plane are AMF and SMF. These NFs are responsible for Packet Data Unit (PDU) session management, such as initial attach of the UE, mobility management or setting up radio bearers. \cite{rommer20195g}
  
The AMF deserves special notice, as it relays all control signaling messages from other CN NFs towards RAN and the UE via the N2 and N1 interfaces. The AMF allows UEs to register, authenticate themselves, and, in coordination with RANs and NFs, ensures smooth handover between cells \cite{rommer20195g}. Although RAN is not extensively discussed in this thesis, it is beneficial to establish how RAN and CN interact to set up communication channels. The AMF provides the RAN with Quality of Service (QoS) parameters, which are used by the Radio Resource Control (RRC) protocol to set up appropriate radio bearers, i.e., logical radio connections between UEs and RAN \cite{dahlman-2020-5g}.
  
To understand the MEC integration into the 5G system, it is necessary to describe PDU session management. The PDU management provides connectivity between UEs and DNs (represented mostly by the Internet). To set up this connection, a UE sends a PDU session establishment request to an AMF, which instructs a SMF to properly register the PDU session, set up (R)AN and create a UPF tunnel towards the DN. The simplified PDU session establishment process is illustrated in Figure~\ref{F:PDU-est}. 5G system architecture allows the UE to be connected to multiple DNs at a time. One PDU session can connect the UE to the Internet, while another one can be set up to connect it with, e.g., local edge server. To determine the correct endpoint DN for a given traffic, UPF uses an Uplink Classifier (Ul CL), which enables steering traffic towards appropriate DNs based on, e.g., uplink IP address. Configuring UL CL can be requested by AF as a SMF service. \cite{rommer20195g}

\begin{figure}[ht]
	\centering
	\includegraphics[width=\textwidth]{./images/PDU-sesh-est.png}
	\caption{PDU session establishment \cite{rommer20195g}.}
	\label{F:PDU-est}
\end{figure}

The PDU session management has seen a minor innovation in 5G, as it introduces completely new workflows: service registration \& discovery and capabilities exposure. These workflows are crucial for understanding how the 5G system interacts with external applications such as MEC. Service registration \& discovery, illustrated in Figure~\ref{F:service-reg-disc}, describes how individual NFs within 5G system utilize the SBA to provide and consume services. Put simply, every service producer needs to register its services with NRF. When a service consumer wants to request services of a particular NF, it first requests NRF for a list of available NFs offering those services. The other workflow, capabilities exposure, describes specifically how external applications can interact with and influence the 5G system. The NFs crucial for this workflow are AF, representing an external application, and NEF, acting as a gateway between external applications and 5G system. Provided that the external application is properly registered, NEF can allow AF to influence traffic steering to forward traffic to a specific DN. This is of a particular interest for MEC integration, since this workflow can allow MEC to influence traffic steering from the UE to edge server. This flow is illustrated in Figure~\ref{F:traffic-control}. \cite{rommer20195g}

\begin{figure}[ht]
	\centering
	\includegraphics[width=\textwidth]{./images/5G-reg-disc.png}
	\caption{Service registration in discovery in 5G \cite{rommer20195g}.}
	\label{F:service-reg-disc}
\end{figure}

\begin{figure}[ht]
	\centering
	\includegraphics[width=15cm]{./images/traffic-control.png}
	\caption{AF influenced traffic control \cite{rommer20195g}.}
	\label{F:traffic-control}
\end{figure}

\section{MEC system concept and architecture}
After providing general overview of the 5G system, as well as identifying individual workflows supporting MEC integration, a general overview of key functional entities (FEs) and workflows that play a role in connecting the MEC system to a 5G mobile network are described in this section.

The general idea of MEC is to bring CC capabilities to the network edge to provide low latency access to computing resources. The low communication latency improves the experience of mobile users, enables emerging applications in industrial IoT or automotive sectors \cite{ETSI:wp36} and opens the 5G ecosystem to new vertical sectors and business opportunities \cite{sabella-mec-sw-dev}.  MEC standards then play an important role in creating a common platform for Mobile Network Operators (MNOs) and MEC application developers from various verticals to aid the development of end-to-end MEC solutions. Therefore, the MEC system framework and reference architecture developed by ETSI ISG MEC (the leading standardization body in MEC) is described.

The MEC framework, defined in \cite{ETSI:GS:MEC003}, is illustrated in Figure~\ref{F:MEC-fw}, which shows the general entities involved in the system, host and network levels. The network level entities ensure connectivity to local area networks, cellular networks and external networks, e.g., the Internet. The host level, consisting of MEC host and its management, is the core of the MEC system, because it represents the infrastructure on which MEC applications (MEC apps) are run. The MEC host can be further dissected into MEC platform, virtualization infrastructure and MEC applications \cite{ETSI:GS:MEC003}. The system level is responsible for management and orchestration of the whole MEC system, typically comprised of multiple MEC hosts \cite{ETSI:GS:MEC003,sabella-mec-sw-dev}. 

\begin{figure}[ht]
	\centering
	\includegraphics[width=12cm]{./images/MEC-framework.png}
	\caption{ETSI MEC framework \cite{ETSI:GS:MEC003}.}
	\label{F:MEC-fw}
\end{figure}

The reference architecture, illustrated in Figure~\ref{F:MEC-ref-arch}, offers a more detailed insight into the general framework (Figure~\ref{F:MEC-fw}) and identifies reference points for connecting individual FEs of the MEC system. The reference points can be divided into three groups based on \cite{ETSI:GS:MEC003}: 
\begin{itemize}
	\item MEC platform functionality reference points (Mp) 
	\item Management reference points (Mm)
	\item Reference points connecting to external entities (Mx)
\end{itemize}
Not all the illustrated reference points are specified by ETSI ISG MEC \cite{ETSI:GS:MEC003}. An example of ETSI ISG MEC specified interface is the Mp1, connecting MEC Platform and MEC apps. This is desirable since it is expected that MEC platform and MEC apps will be developed by different parties. An example of a closed (internal) interface is the Mm5, connecting MEC platform and its management entity, which are both expected to be developed by the same organization and are therefore not specified but left as an implementation choice \cite{sabella-mec-sw-dev,ETSI:GS:MEC003}. 

\begin{figure}[ht]
	\centering
	\includegraphics[width=\textwidth]{./images/MEC-ref-arch.pdf}
	\caption{ETSI MEC reference architecture \cite{ETSI:GS:MEC003}.}
	\label{F:MEC-ref-arch}
\end{figure}

The MEC platform is a crucial MEC system FE, as it acts as an intermediary between MEC apps and the system level management. Through the Mp1 interface, MEC platform offers an environment where MEC apps discover, advertise, consume and offer MEC services \cite{ETSI:GS:app-ena}. This is very similar to the SBA based workflows in 5G system. MEC platform offers the MEC apps similar functionality to what NEF and NRF offer to the 5G system. On top of that, the MEC platform has another important responsibility. Upon request from a MEC app, MEC platform can activate, deactivate or update traffic rules and then configure the data plane accordingly \cite{ETSI:GS:MEC003}. This is analogous to AF functionality within 5G system and elaborated in the next section focusing on integration of MEC system to 5G mobile network.

Another important part of the MEC system is its management - split into host level and system level, which is illustrated in Figure~\ref{F:MEC-fw}. The individual FEs taking part in MEC management are illustrated in Figure~\ref{F:MEC-ref-arch}. System level management is comprised of Operations Support System (OSS), User Application Life Cycle Management Proxy (UALCMP) and MEC orchestrator (MEO). MEO is a key FE overseeing the whole MEC system in terms of deployed MEC hosts, available resources and services, etc. It also plays a key role in handling user mobility or application onboarding, i.e., ensuring the application can run on the MEC system. Host level management responsibilities are shared between two entities: MEC Platform Manager (MEPM) and Virtualization Infrastructure Manager (VIM). MEPM manages the lifecycle of applications and informs MEO about MEC app related events, while VIM manages the virtualized resources of the virtualization infrastructure. \cite{ETSI:GS:MEC003}

Thus far, it has been described how MEC apps and MEC platform interact with each other to provide and consume services, and how the whole MEC system is managed on both system and host levels. The last important workflow to be described is how the MEC system interacts with the UE, so that the UE can discover and exploit available MEC servers.  There are two recognized approaches for edge discovery from the UE perspective: DNS based and device based \cite{ETSI:wp36}. The first option, DNS based, is called edge unaware. Since the client application (application instance in UE) has no information about the edge server location, it is connected to the MEC app via a DNS server. Thus, the UE sends queries to a named server, while these queries can be enriched by the 5G CN with the device location information \cite{ETSI:wp36,ETSI:wp20}. The second approach, device based with edge awareness, requires a new entity on the client side. This entity (device application) communicates with the MEC system to receive the MEC application address and offers it to the client applications \cite{ETSI:wp20}. The edge aware approach is more suited for highly distributed edge cloud exposed to mobile UE, as it allows informing the device application of edge server changes \cite{ETSI:wp36}. In applications that need to keep track of current state, called stateful applications, this requires the MEO to trigger context migration to another server \cite{ETSI:GS:MEC003,ETSI:wp36}. 

\section{Integration of MEC into 5G system}
The previous sections of this chapter provided a brief overview of both 5G system and MEC system with focus on the techniques and concepts that can be used for their seamless integration. This section aims to deliver an end-to-end framework for MEC and 5G integration, based on standards by 3GPP and ETSI MEC. This is of vital importance for the rest of the thesis, but it is also valuable for developers of MEC solutions.

The lifecycle of a MEC app can be divided into four stages, as done in: \cite{ETSI:wp20}:
\begin{enumerate}
	\item MEC app packaging and onboarding
	\item MEC app instantiation 
	\item Client-side app and MEC app communication 
	\item Usage of the MEC platform and services 
\end{enumerate}
MEC apps are expected to be run within a virtualized environment of the MEC hosts, i.e., Virtual Machines (VMs) or containers. The onboarding of MEC apps is left as an implementation choice for the MEC operator \cite{ETSI:wp20}. The second stage of a MEC app lifecycle describes its instantiation on MEC system and its exposure to clients. There are two options for MEC app initialization and of communication establishment between the client-side app and the MEC app. One option is through UALCMP, and the other one is via the OSS without the need for any client-side action. The first option is related to edge-aware discovery (described in the previous section). In UALCMP option, the device application contacts the UALCMP via the Mx2 interface and informs it which MEC app it would like to instantiate, the UALCMP then informs the MEO, which is responsible for choosing the appropriate MEC host and configuring the MEPM to initialize the given application. In response to successful instantiation, the device app receives the IP address of the running MEC app instance \cite{ETSI:wp20}. In the other option, the initialization is done without the client-side being required to participate in the process. For the client app, to leverage MEC, it accesses the required application via its domain name and the 5G system DNS is responsible for connecting the client to the nearest running MEC app instance \cite{ETSI:wp20,ETSI:wp36}. Once instantiated, the MEC apps can leverage SBA of the MEC system, consuming services provided by the MEC platform or other applications and also offering their services to other applications.

The MEC can be exploited in two ways, both providing UE low latency access to the desired application in an edge server. The first option relies on DNS servers directing clients to the nearest edge server. This option requires low implementation effort but has several limitations. It is less suited for highly mobile scenarios (vehicles, users in trains, etc.) and highly distributed cloud resources \cite{ETSI:wp36}. The second option relies on tight integration of the MEC system and the underlying mobile network. As a result, it offers more support for high mobility scenarios and even allows context migration between MEC app instances in stateful applications \cite{ETSI:wp36}.

Furthermore, MEC offers a framework for MEC apps, through which they can feed traffic rules to the mobile network to help satisfy user application QoS requirements. This is achieved through establishment of a dedicated PDU session, steering traffic from the client app to the MEC host.

The framework, utilizing edge aware app initialization and dedicated PDU session establishment is described below and illustrated in Figure~\ref{F:MEC-5G-integration}.
\begin{enumerate}
	\item Device app requests initialization of a MEC app from UALCMP
	\item MEO chooses the optimal MEC host and sends traffic rules to the selected MEC platform
	\item MEC platform, acting as an AF, requests traffic rules control from SMF, via NEF and then PCF 
	\item SMF configures the UPF with UL CL to steer desired traffic to MEC host
\end{enumerate}

\begin{figure}[ht]
	\centering
	\includegraphics[width=\textwidth]{./images/MEC-5G-integration-flowchart.png}
	\caption{MEC integration in 5G.}
	\label{F:MEC-5G-integration}
\end{figure}

%-----------------------
%  Ch3- OS MEC projects
%-----------------------
\chapter{Open-source MEC Solutions}
\label{Ch:o-sMEC}
This chapter provides a comprehensive overview of the expanding ecosystem of open-source MEC solutions with the primary goal of selecting the most suitable solution to leverage in the development of VEC testbed. The emergence of open-source MEC solutions stems from the necessity to accelerate the development of MEC applications and promote the adoption of MEC as a critical technology in 5G mobile networks. Consequently, this environment comprises of various MEC projects, developed by a range of stakeholders, addressing different challenges within MEC. This chapter intends to describe the open-source MEC ecosystem by introducing the main bodies taking part in MEC development, their respective projects and the challenges these projects address.

The main selection criteria for evaluating MEC projects are which part of the ETSI MEC system is covered and the underlying technologies the solution is based on. A critical factor is the interoperability support, i.e., which interfaces are provided in the project to enable communication with modules developed in the rest of the MEC ecosystem. An end-to-end MEC solution is expected to be developed by three stakeholders: MEC app developer, mobile network operator and MEC operator. The evaluated open-source projects play the role of a MEC operator. Therefore, it is crucial that they provide means of integration with both MEC apps and mobile networks. Furthermore, all the evaluated projects are a work in progress, as they do not yet provide all the desired functionalities or are fully operational. Therefore, some promising projects were tested and their state of operationality is discussed.

The main selection criteria for evaluating MEC projects are which part of the ETSI MEC system is covered and the underlying technologies the solution is based on. A critical factor is the interoperability support, i.e., which interfaces are provided in the project to enable communication with modules developed in the rest of the MEC ecosystem. An end-to-end MEC solution is expected to be developed by three stakeholders: MEC app developer, mobile network operator and MEC operator. The evaluated open-source projects play the role of a MEC operator. Therefore, it is crucial that they provide means of integration with both MEC apps and mobile networks. Furthermore, all the evaluated projects are a work in progress, as they do not yet provide all the desired functionalities or are fully operational. Therefore, some promising projects were tested and their state of operationality is discussed.

Some popular hypervisors are VirtualBox or VMware and the most common platform for container virtualization is Docker. Orchestrating multiple containers on a single node is often handled by Kubernetes (k8s), a popular open-source container orchestration tool. Another key software, Open vSwitch, provides virtual networking layer for VMs or containers and enables their connectivity to each other and to data networks. 

In the rest of this chapter, MEC projects are described.

\section{Akraino projects}
The leader in development of open-source edge software is Linux Foundation (LF) Edge, a LF umbrella organization, which aims to unite industry leaders to introduce edge computing framework. LF Edge identifies three levels of maturity within their projects. Only two projects are, at the time of writing, listed in the top category: Akraino and EdgeX Foundry \cite{lf-edge-web}. The latter, EdgeX Foundry, has established itself as the leading open-source industrial IoT edge framework. Akraino, on the other hand, spans over a broad variety of use cases, including but not limited to MEC. Akraino is not one common platform, but rather a whole set of edge stack solutions, called blueprints. Furthermore, in 2019 LF Edge signed a cooperation agreement with ETSI, which lead to an increasing number of Akraino blueprints tightly following ETSI ISG MEC specifications \cite{sabella-mec-sw-dev}. Under the newest – sixth release, Akraino constitutes of over 30 blueprints, such as: Connected Vehicles Blueprint (CVB), Enterprise Applications on Lightweight 5G Telco Edge (EALTEdge) or Public Cloud Edge Interface (PCEI) \cite{lf-edge-web}. 
\subsection{Connected Vehicles Blueprint}
The CVB blueprint is developed by Tencent, Arm, Intel and Nokia. It aims to create a MEC platform offering services to connected vehicles. Some applications running on top of the provided platform include traffic situation monitoring, road event perception or high precision positioning. It references the ETSI MEC Mp1 and Mm5 interfaces, which are supposed to handle communication between MEC platform, MEC platform management and MEC apps. It is not uncommon among MEC solutions to leverage some related open-source software, CVB uses StarlingX as a virtualization platform, TARS as a microservices management tool and OpenNESS toolkit for managing and onboarding edge services. The setup of CVB requires three nodes: Jenkins master, TARS master and TARS node with the vehicle applications. An illustration of the scope of CVB is shown in Figure~\ref{F:cvb}. \cite{cvb-docu} 
\begin{figure}[ht]
	\centering
	\includegraphics[width=\textwidth]{./images/akraino-cvb.png}
	\caption{Technical overview of CVB \cite{cvb-docu}.}
	\label{F:cvb}
\end{figure}

\subsection{Enterprise Applications on Lightweight 5G Telco Edge}
EALTEdge is a blueprint under 5G/MEC system blueprint family within the Akraino project. It is intended to play the role of a MEC operator through offering MEC platform as well as its management and orchestration to host enterprise applications on lightweight 5G telco edge. EALTEdge uses EdgeGallery as its upstream project, which provides edge platform, application management and platform for application developers. On top of EdgeGallery, EALTEdge intends to integrate other Akraino blueprints, such as EdgeConnector and EdgeGateway, to create a richer environment with even more capabilities. EdgeGallery provides several ETSI MEC APIs such as application enablement API as defined in \cite{ETSI:GS:app-ena}, it is also one of the projects listed in \cite{mec-etsi-web} as relevant MEC solutions. The general architecture of EALTEdge project, leveraging EdgeGallery is provided in Figure~\ref{F:ealtedge}. \cite{ealtedge-docu} 
\begin{figure}[ht]
	\centering
	\includegraphics[width=13cm]{./images/akraino-ealtedge.png}
	\caption{Technical overview of EALTEdge \cite{ealtedge-docu}.}
	\label{F:ealtedge}
\end{figure}

The deployment of EALTEdge is quite flexible, allowing to deploy everything containerized in one node (bare metal or virtual) or it can be deployed in multiple (typically two) nodes. Furthermore, the deployment is managed through provided ansible playbooks to simplify the whole process.

Because EALTEdge provides a general MEC system, which closely follows ETSI MEC specifications and also offers a development environment for MEC apps, it seems like an ideal platform for the testbed development. However, upon testing in lab environment, it was discovered that EdgeGallery servers forbid requests of downloading the package containing the software for the user apps. Therefore, while EALTEdge or even EdgeGallery alone seem like very promising projects, they could not be leveraged in testbed development.

\section{Intel Edge Projects}
Beside the Akraino project, another primary mover within edge software development is Intel. Intel develops commercial end-to-end edge solutions for their customers, but they also offer open-source variants for developers to prototype on. The range of edge-oriented products currently offered by Intel and the licenses under which they are released are shown in Figure~\ref{F:intel-solu}. A successful open-source solution by Intel was Open Network Edge Services Software (OpenNESS) \cite{sabella-mec-sw-dev}, which at some point gained a lot of momentum within MEC open-source community. Moreover, several Akraino blueprints reference OpenNESS as a platform they were initially built on \cite{cvb-docu,akraino-mec-slice-docu}. However, the OpenNESS project was shut down and transitioned towards Smart Edge Open. Both platforms are introduced in the following two subsections.
\begin{figure}[ht]
	\centering
	\includegraphics[width=13cm]{./images/intel-solu.png} 
	\caption{Intel edge solutions developed under different licenses \cite{seo-overview}.}
	\label{F:intel-solu}
\end{figure}

\subsection{OpenNESS}
OpenNESS is a software toolkit that enables building custom platforms for edge computing use cases, including MEC, and are capable of onboarding and managing applications. Although OpenNESS does not strictly follow the ETSI MEC specifications, it was inspired by them and also references 3GPP and O-RAN specifications that were used to develop APIs for integration with 5G mobile networks \cite{sabella-mec-sw-dev}. It should be noted that Intel offered its own solution for 5G CN emulation, compatible with OpenNESS, though not as a part of the open-source package but as a commercial product \cite{openness-git}.

OpenNESS was built on top of Kubernetes and highly optimized for edge computing. A typical deployment consists of an OpenNESS Kubernetes Edge Node and an OpenNESS Kubernetes Control plane, as shown in Figure~\ref{F:openness}. \cite{openness-git}
\begin{figure}[ht]
	\centering
	\includegraphics[width=13cm]{./images/openness.png} 
	\caption{Technical overview of OpenNESS \cite{openness-git}.}
	\label{F:openness}
\end{figure}

\subsection{Smart Edge Open}
Smart Edge Open (SEO) is a result of Intel’s efforts to unify development of its open-source and commercial products. SEO follows the basics of OpenNESS architecture, as it is also built on top of Kubernetes and the basic deployment model consists of a control node and an edge node \cite{seo-doc}. The resulting architecture is shown in Figure~\ref{F:seo}. Even though SEO development initially started as a rebranding of OpenNESS, the documentation of SEO is not as rich and it further distances itself from the ETSI MEC architecture. Eventually, while the transition from OpenNESS to SEO may have improved the platform’s integration into Intel ecosystem, it distances itself from open-source MEC community, where strict following of standardized architecture is necessary to ensure interoperability between individual projects. 

Even though it was argued that due to its distancing from strict ETSI MEC reference architecture, SEO is not an optimal solution for testbed development, it was still tested in our lab environment. However, the deployment of the platform was unsuccessful. SEO uses Intel’s own provisioning software and while it should simplify the deployment, it makes the process of troubleshooting unfeasible, when an issue is encountered.
\begin{figure}[ht]
	\centering
	\includegraphics[width=13cm]{./images/intel-seo.png} 
	\caption{Technical overview of SEO \cite{seo-doc}.}
	\label{F:seo}
\end{figure}

\section{EURECOM and OpenAirInterface Projects}
The two following projects have been developed by EURECOM and are closely related to OpenAirInterface (OAI), the mobile network emulation platform that is now maintained by OAI Software Alliance (OSA). EURECOM though continues to be involved in the development OAI and plays a key role in expanding the scope of OAI to network slicing or MEC.
\subsection{LL-MEC}
Low latency MEC (LL-MEC) was introduced in 2018 as the first open-source low latency MEC platform, enabling mobile network monitoring, control and programmability \cite{nikaein2018llmec}. While it was developed in compliance with 3GPP and ETSI specifications, these specifications are not related to 5G MEC integration referenced in Chapter~\ref{Ch:MECn5G}, since LL-MEC was developed as a 4G MEC solution. One possible deployment of LL-MEC is shown in Figure~\ref{F:llmec}, which also demonstrates that LL-MEC uses Open vSwitch for user plane. Since this thesis puts main focus on integrating MEC in 5G mobile networks, the LL-MEC has not been chosen as the platform for testbed development.
\begin{figure}[ht]
	\centering
	\includegraphics[width=13cm]{./images/ll-mec.png} 
	\caption{Technical overview of LL-MEC \cite{nikaein2018llmec}.}
	\label{F:llmec}
\end{figure}

\subsection{MEC platform by EURECOM}
This MEC Platform (MEP) has been only recently developed by EURECOM and is closely integrated with OAI emulated 5G network, therefore, the rest of the text refers to it as OAI MEP. At the time of writing, the OAI MEP is very much a work in progress as it only implements a MEP environment for registry and discovery of MEC services and offers its own implementation of Radio Network Information Service (RNIS) \cite{oai-mep-git}. However, it has several traits that make it a very viable candidate for VEC testbed development. Its integration with the OAI ecosystem would significantly reduce the integration implementation overhead. Moreover, it promises to follow ETSI MEC as well as O-RAN specifications. And finally, the developers have already outlined that it is intended to implement traffic rules control via communicating with PCF in 5G CN. This would mean that OAI MEP is aiming for a MEC implementation very close to the standardized approach extensively described in Chapter~\ref{Ch:MECn5G}. The components thus far developed are shown in Figure~\ref{F:oai-mep}.
\begin{figure}[ht]
	\centering
	\includegraphics[width=13cm]{./images/OAI-MEP.png}
	\caption{Technical overview of OAI-MEP \cite{oai-mep-git}.}
	\label{F:oai-mep}
\end{figure}

\section{Other MEC projects}
\subsection{AdvantEDGE}
AdvantEDGE is a mobile edge emulation platform running on Docker and Kubernetes. Its main ambition is to help MEC app developers decide where in the network edge their application should run and how different network characteristics such as latency, packet loss or jitter impact their application’s QoS. This is achieved through emulation of mobile network with the possibility to test the application with different network characteristics settings and different deployment models (how close to the user the app is run). On top of that, the platform enables triggering of mobility events and application relocation even for stateful applications. The platform is not intended to emulate any part of the ETSI MEC reference architecture, but it implements various ETSI MEC APIs, such as RNIS, Location Service or App Enablement Service. The high-level overview of AdvantEDGE micro-service based architecture is shown in Figure~\ref{F:advantedge}. \cite{advantedge-docu}
\begin{figure}[ht]
	\centering
	\includegraphics[width=13cm]{./images/advatnedge.png} 
	\caption{Technical overview of AdvantEDGE \cite{advantedge-docu}}
	\label{F:advantedge}
\end{figure}

AdvantEDGE was successfully deployed in our lab setup, however, it was deemed not suitable for the purposes of the testbed development, which puts a strong emphasis on the platform architecture and the underlying mobile network.

In total, seven open-source projects have been examined. CVB by Akraino has not been chosen due to difficulties encountered when trying to modify the code. EALTEdge could not be deployed, as well as, for various reasons, OpenNESS and SEO. LL-MEC is not a perspective solution for 5G networks. OAI-MEP has been deployed and tested successfully, it follows the ETSI MEC specifications and while its capabilities are as of yet limited, plans to extend the platform’s features have already been outlined. Therefore, OAI-MEP has been chosen as the platform for VEC testbed development. Simplified evaluation of the examined projects is provided in Table~\ref{T:os-mec-proj}.
\begin{table}[!ht]
    \centering
	\caption{Evaluation of examined open-source projects}
	\label{T:os-mec-proj}
    \begin{tabular}{|l|l|l|l|}
    \hline
        & ETSI MEC reference & open-source & \Longunderstack{successfully\\deployed} \\ \hline
        CVB & MEC platfrom & yes & no \\ \hline
        EALTEdge & MEC system & yes & no \\ \hline
        OpenNESS & MEC plaform + platform management & partially & no \\ \hline
        SEO & edge server and management & partially & no \\ \hline
        LL-MEC & 4G MEC solution & yes & no \\ \hline
        OAI MEP & MEC platform & yes & yes \\ \hline
        AdvantEDGE & MEC APIs (RNIS) & yes & yes \\ \hline
    \end{tabular}
\end{table}
%-------------------------
%  Ch4- OpenAirInterface
%-------------------------
\chapter{Mobile network emulation} % to include OAI in chapter title  \texorpdfstring{\\ \mbox{OpenAirInterface}}{OpenAirInterface}
\label{Ch:OAI}
This chapter provides an overview of an open-source mobile network emulation platform Open Air Interface (OAI). OAI has the ambition to democratize the research and development of wireless access technologies through delivering a 3GPP compliant software stack for both RAN and CN. The OAI software offers different levels of mobile networks simulation and emulation in flexible deployment scenarios and is intended to run on general purpose hardware \cite{kaltenberger-oai}. The development of the OAI has begun at EURECOM, before being passed to OAI Software Alliance (OSA), which has the goal of promoting the adoption of OAI across both academia and industry.

OAI development is split into different projects. The two biggest ones are focused on developing the solutions for RAN and CN. While those are definitely the core of OAI, its focus spans further and exploits smart solutions regarding mobile networks’ orchestration and programmability under Mosaic5G project. The following sections offer a brief introduction to all of these projects and discuss how they are going to be leveraged in OAI based VEC testbed.
%
\section{OAI RAN}
OAI RAN develops and implements code for running 4G and 5G base station (eNB in 4G and gNB in 5G) and User Equipment (UE). These implementations are developed with respect to 3GPP specifications and therefore, OAI RAN implements the entire 5G New Radio (5G-NR) protocol stack \cite{oai_doc}. This thesis is focused primarily on the CN architecture, therefore, the individual protocols of 5G-NR are provided in a detail needed for the purpose of this thesis. In Figure~\ref{F:5G-protstack} is an example of 5G-NR user plane protocol stack with brief introduction of individual layers based on \cite{dahlman-2020-5g}.
%
\begin{figure}[ht]
	\centering
	\includegraphics[width=13cm]{./images/5G-NR-protstack.png} 
	\caption{5G-NR user plane protocol stack \cite{dahlman-2020-5g}.}
	\label{F:5G-protstack}
\end{figure}
%
The three topmost layers are the application layer, the Transmission Control Protocol (TCP) and Internet Protocol (IP). This is common for any IP-based network, including 4G and 5G mobile networks.

The 5G NR user plane protocol stack:
\begin{itemize}
	\item Service Data Application Protocol (SDAP): responsible for mapping of QoS bearers to radio bearers, new layer in NR.
	\item Packet Data Convergence Protocol (PDCP): performs IP header compression, ciphering and integrity protection.
	\item Radio Link Control (RLC): segmentation and retransmission handling.
	\item Medium Access Control (MAC): multiplexing of logical channels, hybrid retransmission and scheduling.
	\item Physical Layer (PHY): coding and decoding, modulation and demodulation or multi-antenna mapping.
\end{itemize}
In control plane, the lower-level protocols are the same, but on top of PDCP there is RRC and NAS.
\begin{itemize}
	\item Radio Resource Control (RRC): setup and release of radio bearers, mobility management, CP signaling.
	\item Non-Access Stratum (NAS): signaling messages between UE and CN for management of UE mobility and network selection.
\end{itemize}
%
For a better understanding of current OAI 5G NR capabilities, some general features based on \cite{oai_doc} are listed below in Table~\ref{T:OAI-featureset}. Note that some of the values, such as downlink throughput, are achieved in an optimized deployment in OAI lab and in general deployment are lower.
%
\begin{table}[!ht]
    \centering
    \caption{OAI RAN features}
    \begin{tabular}{|l|l|}
    \hline
        Downlink Throughput & 1000 Mb/s \\ \hline
        Uplink Throughput & 100 Mb/s \\ \hline
        Round Trip Time (RTT) & 4 ms \\ \hline
        Number of UEs & 16 \\ \hline
        Duplexing & TDD, FDD \\ \hline
        Bandwidths & 10, 20, 40, 80, 100 MHz \\ \hline
        Frequency Ranges (FRs) & FR1, FR2 \\ \hline
        Slot format & 14 OFDM symbols \\ \hline
        Subcarrier modulation & up to 256 QAM \\ \hline
        MIMO support & yes, in downlink \\ \hline
    \end{tabular}
    \label{T:OAI-featureset}
\end{table}
%
OAI is designed to serve as a development tool for accurate mobile network simulations and emulations. For this purpose, OAI enables different simulation models: RFsimulator, L2 nFAPI simulator and running with true radio heads. RFsimulator replaces the radio heads, which in emulation run on Software Defined Radios (SDRs) by software implementation. This allows a complete 4G and 5G setup on a generic PC. In this case, virtual interfaces are created in the PC to enable sending real traffic. It is the optimal model to start testing OAI as it follows the same protocols as if running with the radio interface with SDR. If the SDR is exploited for emulation, the same code is run, which greatly decreases development time. The throughput achieved with RFsimulator depends on the host machine’s CPU power. L2 nFAPI simulator allows connecting the UEs directly to eNB or gNB by bypassing Physical layer (PHY) via nFAPI (Network Functional API), which extends split between PHY and MAC \cite{scf-nfapi}. The third option, mobile network emulation, is designed to operate with true radio interface, represented by a SDR such as USRP (Universal Software Radio Peripheral) to enable connecting Customer-Off-The-Shelf (COTS) UE, i.e., publicly available smarthpones \cite{oai_doc}. It should be noted that at the time of writing, only a limited number of UEs (smartphones) work with OAI 5G network and it requires a lot of configuring \cite{oai-maillist}.

The development of OAI 5G CN is an ongoing process, while evolution from LTE in 4G to NR in 5G seems to be complete with OAI enabling standalone 5G deployment, OSA defined the short-term goals of OAI as performance improvement or integration with smart RAN solutions developed within the Mosaic5G group. \cite{oai-mosaic5g-web}
%
\section{OAI CN}
OAI 5G CN project group develops open-source solution for standalone 3GPP compliant 5G CN that is capable of enabling various 5G use cases. The 5G CN components are continuously tested with open-source as well as commercial RAN solutions \cite{oai-cn-web}. The OAI 5G CN has been gradually developed from 4G EPC to the 5G CN architecture, extensively described in Chapter~\ref{Ch:MECn5G}. In fact, this work is still ongoing, with the current state-of-the-art shown in Figure~\ref{F:oai-cn-soa}.
%
\begin{figure}[ht]
	\centering
	\includegraphics[width=13cm]{./images/OAI-CN-soa.png} 
	\caption{OAI 5G CN project state-of-the-art \cite{oai-cn-doc}}
	\label{F:oai-cn-soa}
\end{figure}
%
According to \cite{oai-cn-web}, OAI 5G CN currently supports many of the key workflows described in Chapter~\ref{Ch:MECn5G}, such as UE connection, registration and session management. Network slicing is already partially supported, but a key interface for opening the CN to third parties, the AF, is not yet available. This is important from the perspective of integrating an open-source MEC system solution with OAI CN, because AF should play a key role in providing the means of accessing the CN’s SBA for MEC orchestrator or MEC platform. 

The flexibility of OAI 5G CN design allows for different deployment models based on deployed NFs. In the minimalist example supporting basic session management procedures, AMF, SMF, NRF and UPF comprise the whole CN. The basic deployment model consists of the minimalist model and AUSF, UDM and UDR \cite{oai-cn-web}. UDR (Unified Data Repository) was not described in Chapter~\ref{Ch:MECn5G}, it is a NF providing storage and retrieval of structured data for PCF, NEF and UDM \cite{ETSI:TS:udr}. Finally, the slicing deployment model utilizes the basic model with NSSF \cite{oai-cn-web}.

As was the case with OAI 5G RAN project, OAI 5G CN is still being developed to offer new capabilities and enable new use cases. During 2022, OAI released PCF and currently works on integrating PCF induced traffic steering. This is a key part of the MEC integration described in Chapter~\ref{Ch:MECn5G}.
%
\section{Mosaic5G}
Mosaic5G is a project group within OAI focused on transforming the underlying OAI RAN and CN into agile platforms providing ecosystem for use cases ranging from centralized network control to edge deployments \cite{oai-mosaic5g-web}. The primary focus of Mosaic5G project group can be dissected into four main projects as defined in \cite{oai-mosaic5g-web}:
%
\begin{itemize}
	\item The O-RAN E2 protocol – E2 agent
	\item A Flexible RAN Intelligent Controller (FlexRIC)
	\item A Flexible Core Controler (FlexCN)
	\item Higher layer RAN and CN operator – Tirematics
\end{itemize}
Currently, most of the Mosaic5G project group’s work is centered around FlexRIC and its integration with the rest of the OAI software \cite{oai-mosaic5g-web}. In fact, the OAI MEP discussed in previous chapter also utilizes FlexRIC as a key component \cite{oai-mep-git}. 
%-------------------------
%   Ch5 - VEC testbed
%-------------------------
\chapter{VEC testbed}
\label{Ch:VEC-tstbed}
This chapter describes the implementation of testbed for VEC as the primary goal of the thesis. The structure of the following text is as follows: the whole task is dissected into three key stages, then each stage is examined in individual subsequent sections.
%
\begin{itemize}
	\item[\textbf{Stage 1:}] VEC testbed design leveraging chosen open-source projects 
	\item[\textbf{Stage 2:}] creating a blueprint for future VEC applications development
	\item[\textbf{Stage 3:}]  evaluation of set up testbed against cloud deployment
\end{itemize}
%
\section{VEC testbed design}
The final testbed design leverages the OAI MEP, introduced in Chapter~\ref{Ch:o-sMEC}. While there, the focus was primarily on evaluating it against other MEC projects in terms of network architecture, this section describes the individual components from developer’s perspective. OAI MEP does not comprise the whole MEC system, but it provides an ETSI compliant MEP with a service-based environment, fully compatible with the underlying OAI RAN and CN. Such setup is sufficient for experimentation with VEC applications, such as task offloading.

The leveraged platform is a blueprint showcasing OAI MEP with RNIS and it integrates five key projects as shown in Figure~\ref{F:VEC-tstbed-arch} \cite{oai-blueprint-git}:
%
\begin{itemize}
	\item OAI 5G RAN
	\item OAI 5G CN
	\item OAI MEP
	\item OAI Configuration Manager (OAI-CM)
	\item RNIS
\end{itemize}
%
\begin{figure}[ht]
	\centering
	\begin{sideways}
		\includegraphics[width=20cm]{./images/OAI-MEP-parts.png} 	
	\end{sideways}
	\caption{VEC testbed design leveraging OAI MEP bluperint \cite{oai-blueprint-git}.}
	\label{F:VEC-tstbed-arch}
\end{figure}
%
This blueprint showcases how MEC apps can consume services, such as RNIS, via the Mp1 interface \cite{oai-blueprint-git}. For RNIS to be able to consume mobile network Key Performance Indicators (KPIs) via Mp2 and then expose them via Mp1 to MEC apps, the OAI mobile network required new features to be implemented. The OAI 5G CN is run in a basic deployment, as shown in Figure 22, but the RAN leverages an O-RAN xApp and Mosaic5G FlexRIC on top of the modules for UE and gNB. The xApp provides a northbound interface for the RNIS to subscribe to radio related KPIs \cite{oai-rnis-git}. The OAI configuration manager then provides the functionality to subscribe to AMF and SMF events \cite{oai-cm-git}.

A key component of the OAI MEP is the MEP Gateway, which acts as a reverse proxy, routing the Mp1 traffic to the right service \cite{oai-mep-git}. The reverse proxy is implemented leveraging Kong \cite{kong-docu}. It hides the underlying service nodes from the clients and routes client’s requests to the right service nodes. This means that all registered services can be accessed via the MEP Gateway.

The Discovery and Registry Service has the responsibility of exposing REST API for the Mp1 interface allowing the MEC apps to do the following, based on \cite{oai-mep-git}:
%
\begin{itemize}
	\item Discover existing services (GET)
	\item Filter hosted MEC services based on service type (GET)
	\item Register a new service (POST)
	\item Delete a hosted MEC service (DELETE)
\end{itemize}
%
The OAI MEP blueprint delivers docker-compose scripts for deployment of all the containers. Furthermore, the basic deployment uses RFsimulator, although \cite{oai-blueprint-git} claims that it is possible to run with USRP radio heads and that it was tested in the OAI lab. 

After a successful deployment, it should be possible to ping the internet from the UE container via its \verb|oaitun_ue1| interface. In our setup, it looks like this:
\begin{verbatim}
	root@9929963b00c2:/opt/oai-nr-ue# \
	ping -c 3 -I oaitun_ue1 comtel.fel.cvut.cz \
	PING glumhosting.fel.cvut.cz (147.32.210.146) from 12.1.1.2 oaitun_ue1:
	56(84) bytes of data.
	64 bytes from glum-146.feld.cvut.cz (147.32.210.146):
	icmp_seq=1 ttl=59 time=8.64 ms
	64 bytes from glum-146.feld.cvut.cz (147.32.210.146):
	icmp_seq=2 ttl=59 time=5.43 ms
	64 bytes from glum-146.feld.cvut.cz (147.32.210.146):
	icmp_seq=3 ttl=59 time=5.06 ms

	--- glumhosting.fel.cvut.cz ping statistics ---
	3 packets transmitted, 3 received, 0% packet loss, time 2003ms
	rtt min/avg/max/mdev = 5.062/6.382/8.649/1.610 ms
\end{verbatim}

To verify that the RNIS works properly, it should be possible to access its service with the HTTP GET method from the Command Line Interface (CLI). This is the result achieved in our setup:
\begin{verbatim}
	lab@lab-HP-Z4-G4-7-Workstation:~/blueprints/mep/examples$ curl -X \
	'GET' 'http://oai-mep.org/rnis/v2/queries/layer2_meas' -H \
	'accept: application/json'
	[
	{
		"KPIs": {
		"bler_dl": {
			"kpi": "bler_dl",
			"labels": {
			"amf_ue_ngap_id": 1
			},
			"source": "RAN",
			"timestamp": 1683903896587163,
			"unit": null,
			"value": 5.605193857299268e-45
	...
\end{verbatim}

\section{VEC application blueprint}
This stage aims to deliver a blueprint for development of VEC applications running on the OAI MEP and consuming its services. It is split into two subsections, with the first one covering the service registration and the second one demonstrating implementation of a MEC app consuming services offered by the OAI MEP.
%
\subsection{Service registration}
The registration and discovery service of OAI MEP can be accessed via its Graphical User Interface (GUI) on the host \verb |192.168.70.5| or via the MEP Gateway:\\\verb |http://oai-mep.org/service_registry/v1/ui|. It should be noted that for the latter to work, the oai-mep.org domain needs to be added to the local machine DNS, which is documented in \cite{oai-blueprint-git}. The GUI offers the services mentioned in the previous section. After a successful deployment, the discovery of available services should yield the following results, as only the RNIS is registered:
%
\begin{verbatim}
	lab@lab-HP-Z4-G4-7-Workstation:~/blueprints/mep/examples$ curl /
	http://oai-mep.org/service_registry/v1/discover 
	[
	{
		"description": "The service allows other app or services to consume
						network and radio information",
		"endpoints": [
		{
			"method": "GET",
			"name": "rab_info",
			"path": "/queries/rab_info"
		},
		...
		],
		"host": "oai-mep.org",
		"name": "rnis",
		"path": "/rnis/v2",
		"port": "80",
		"sid": "6458bfb3aab751de1b2ba9f0",
		"type": "RadioNetworkInformation"
  },
\end{verbatim}

To demonstrate the service registry and discovery function of OAI MEP, a sample hello-world flask application has been created. This application runs on the localhost interface of the host machine and displays a simple line of text, the goal is to be able to route this application via the MEP Gateway.
%
\begin{flask}[caption={Sample flask app in python}]
	from flask import Flask

	app = Flask(__name__)

	@app.route('/')
	def root():
		return "<h1>VEC testbed root page</h1>"

	@app.route('/hello')
	def hello():
		return '<h1>Hello world page</h1>'

	if __name__ == '__main__':
		app.run(debug=True, host='0.0.0.0', port=5000, use_reloader=False)
\end{flask}

To register the flask application, the GUI of the registry and discovery service has been leveraged. While the GUI provides a sample input for the register service, it does not include any \verb|uid|, which confuses the underlying database of OAI MEP and results in unsuccessful registration. For a successful registration, the following input for the register command has been used:
%
\begin{verbatim}
{
	"description": "testing registration with sample flask",
	"endpoints": [],
	"host": "tstbed-app.org",
	"name": "tstbed-app",
	"path": "/",
	"port": 5000,
	"sid": "sid",
	"uid": "tstbedUID",
	"type": "Traffic"
	}
\end{verbatim}
Successful registration is indicated by the code \verb|201| and a response in \verb|json| format about the newly registered service, then sample flask application can be routed via the MEP Gateway, as demonstrated in Figure~\ref{F:flask-hello-demo} below.
%
\begin{figure}[!ht]
	\centering
	\includegraphics[width=13cm]{./images/flask-demo.png} 
	\caption{Demonstration of successful app registration.}
	\label{F:flask-hello-demo}
\end{figure}
\subsection{MEC app consuming services}
This subsection demonstrates how to create an application that would consume the services offered by OAI MEP based on an example app provided in \cite{oai-blueprint-git}. The example app is a flask application that subscribes to RNIS and shows the received data. The core of the application with minor changes from the version in \cite{oai-blueprint-git} is shown in Code~XYZ.
%
\begin{flask}[caption={example-app.py}]
	...
	# create a subscription for receiving l2 measurements
	sub_endpoint = "http://oai-mep.org/rnis/v2/subscriptions"
	sub_body ={
	"callbackReference": f"http://{ip_addr}:{port}/subscriptions/l2meas-200",
	"filterCriteriaNrMrs": {},
	"subscriptionType": "NrMeasRepUeSubscription",
	"expiryDeadline": {
		"nanoSeconds": 12133423,
		"seconds": 20
	}
	}
	requests.post(url =sub_endpoint, json=sub_body)
	...
	# collect received data
	@app.route('/subscriptions/l2meas-200', methods=[ 'POST'])
	def receive_notification():
		if request.method == 'POST':
			content = request.get_json(force=True)
			print(content)
			kpis = content["Report"]
			for aid in content["AssociateId"]:
				if aid not in collected_data:
					collected_data[aid] = {}
				for kpi_name, kpi_data in kpis.items():
					collected_data[aid][kpi_name] = kpi_data['value']
		return "OK"

	# show kpis
	@app.route('/dashboard', methods=['GET'])
	def dashboard():
		return "<h1> Network Monitoring Dashboard</h1> " + json.dumps(collected_data)
	...
\end{flask}
%
It is noteworthy that the example app uses the reverse proxy OAI MEP Gateway to subscribe to RNIS as shown on line 3:  \\\verb |sub_endpoint = http://oai-mep.org/rnis/v2/subscriptions|. Then in \verb|sub_body|, an endpoint for receiving the data from RNIS is provided, and finally, under that endpoint, the \verb|receive_notification()| function handles the received data and puts them into \verb|collected_data| dictionary.
%
\section{Evaluation of VEC testbed}
To evaluate the VEC testbed, an experiment has been conducted, measuring the RTT between the UE and the OAI MEP host, against the RTT between the UE and a cloud. The diagram illustrating the experiment is shown in Figure~\ref{F:experiment}. It does not account for the UE and gNB modules also being hosted on the edge server, although the experiment could be scaled with using radio heads and both UE and gNB modules in standalone machines. The deployment in our lab setup follows the deployment model collocating edge server (MEC host) with UPF [i]. The cloud was set up using cloud service provider Linode and it was deployed in Frankfurt, to simulate realistic cloud distance from the UE.
\begin{figure}[!ht]
	\centering
	\includegraphics[width=\textwidth]{./images/experiment.png} 
	\caption{Design of experiment.}
	\label{F:experiment}
\end{figure}
%
The experiment consisted of running ping from UE’s \verb|oaitun_ue1| interface (available on the PC with RFsim) to both the edge and cloud. The measurement was issued every two hours throughout one day and the results are shown in Table~\ref{T:ping} and visualized in Graph~\ref{G:pings-scatt}. 

The data show that the average RTT difference between edge and cloud is $7.94 \mbox{ ms}$. This follows the expectation and demonstrates the feasibility of using OAI MEP setup for VEC testbed.
%
\begin{table}[!ht]
    \centering
	\caption{RTT measurement}
	\label{T:ping}
    \begin{tabular}{|l|c|c|}
    \hline
        & \textbf{Edge [ms]} & \textbf{Cloud [ms]} \\ \hline
        1:00 & 5.810 & 13.137 \\ \hline
        3:00 & 5.848 & 13.531 \\ \hline
        5:00 & 5.694 & 13.559 \\ \hline
        7:00 & 5.545 & 14.311 \\ \hline
        9:00 & 5.311 & 13.079 \\ \hline
        11:00 & 5.969 & 13.380 \\ \hline
        13:00 & 5.809 & 13.623 \\ \hline
        15:00 & 5.066 & 13.458 \\ \hline
        17:00 & 5.792 & 13.589 \\ \hline
        19:00 & 5.333 & 13.403 \\ \hline
        21:00 & 5.507 & 13.410 \\ \hline
        23:00 & 5.335 & 13.868 \\ \hline
    \end{tabular}
\end{table}
%
\begin{figure}[ht]
	\centering
	\captionsetup{labelsep=colon,name=Graph}
	\includegraphics[width=13cm]{./images/RTT-comparison.png} 
	\caption{RTT for edge and cloud.}
	\label{G:pings-scatt}
\end{figure}
\begin{figure}[ht]
	\centering
	\captionsetup{labelsep=colon,name=Graph}
	\includegraphics[width=13cm]{./images/RTT_comp_box.png} 
	\caption{RTT for edge and cloud.}
\end{figure}
%
he second experiment evaluated the network performance and available bandwidth with iperf. This experiment deserves special notice. As mentioned in Chapter~\ref{Ch:OAI}, the total achievable network throughput in RFsimulator mode depends on the underlying hardware, the actual numbers are, therefore irrelevant, as they scale with the hardware. The throughputs achieved for both edge and cloud showed no significant difference, as highlihted in Table~\ref{T:throughput}.
\begin{table}[!ht]
    \centering
    \caption{Throughput measurement}
	\label{T:throughput}
    \begin{tabular}{|c|c|c|c|}
    \hline
        \multicolumn{2}{|c|}{Edge} & \multicolumn{2}{c|}{Cloud} \\ \hline
        Downlink [Mbps] & Uplink [Mbps] & Downlink [Mbps] & Uplink [Mbps] \\ \hline
        9,09 & 7,18 & 9,01 & 7,195 \\ \hline
    \end{tabular}
\end{table}
% -----------------------
%     Conclusion
% -----------------------
\chapter{Conclusion}
The thesis’s objective was to set up a functional testbed for future research on VEC based on standards developed for MEC. To facilitate these objectives, the standards regarding MEC and 5G have been examined and a framework for their integration has been proposed. Then, existing open-source solutions for MEC or VEC have been evaluated with the purpose of selecting the most suitable platform for leveraging in VEC testbed. The selected platform has been thoroughly examined, tested and evaluated against realistic cloud deployment scenario. 

The set-up testbed leverages the OAI MEP project. It is built on OAI 5G RAN and CN with additional components that ensure exposing underlying mobile network information to implemented RNIS. The MEP itself acts as a gateway for MEC apps to consume MEP registered services. The 5G NFs and MEC FEs in OAI MEP run as Docker containers in RFsimulator simulation model with the possibility of extending the simulation to use COTS hardware for the radio interface. To enable seamless extension of the testbed and future work, the stable OAI MEP code has been forked to a new VEC testbed repository and a framework for registering new MEC service and consuming registered services has been provided. 

The testbed has been evaluated against cloud deployment to assess the achieved reduction in communication latency. The edge deployment model employed in our lab has the edge server collocated with UPF, emulating a common approach. The evaluation results clearly demonstrate a significant decrease in RTT, approximately 59 \% when utilizing an edge deployment compared to the cloud alternative. This establishes the testbed’s viability as a research tool to evaluate future VEC applications. 

In future, the testbed should be updated with future releases of the underlying OAI MEP, which promises to further increase the mobile network programmability through MEP inflicted traffic control via PCF. Furthermore, the testbed could be expanded to work with real radio interface to provide even more accurate evaluations. The platform now enables development of VEC applications leveraging task offloading for various use cases such as remote image processing for AVs. With research towards the 6G having already started, the VEC and in general edge computing is envisioned to be extended towards Non-Terrestrial Networks, i.e., providing edge computing services in the air (low and high-altitude platforms) or space (satellites). This is already a topic of 5G, but its primary deployment will be seen in the 6G, similar to MEC in 4G.

% -----------------------
%     BIBLIOGRAPHY
% -----------------------
\bibliographystyle{IEEEtran}
%\addcontentsline{toc}{chapter}{References}
\bibliography{thesis}


\clearpage
%\bibliographystyleNew{IEEEtran}

%
%\begin{appendices}
% \chapter{TPS project requirements}
% This project also requires the use of an equation and a graph. I decided to append that to the end of this work, since I could not think of any suitable equation or graph for the project.
% Graph \ref{only-Graph} shows a sine function with added gaussian noise. Equation \ref{Eq:Shan-H} is a Shanon Hartley theorem.

%\captionsetup[figure]{name=Graph}
%\begin{figure}[h]
%	\includegraphics[width=\textwidth]{fig1.pdf}
%	\caption{Sine function with gaussian noise}
%	\label{only-Graph}	
%\end{figure}
%
%Shanon-Hartley theorem:
%\begin{equation}
%	C = B \log_2 \left(1 + \frac{S}{N}\right)
%	\label{Eq:Shan-H}
%\end{equation}
%$C$ is the capacity in bits per second, $B$ is bandwith in Hz, $S$ and $N$ are powers of a signal and noise respectively.
%
%\end{appendices}
	
	
\end{document}